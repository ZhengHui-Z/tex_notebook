\documentclass{beamer}

%!%!%!%!%!%!%!%!%
%! dependencies %
%!%!%!%!%!%!%!%!%

%! font encoding
\usepackage[utf8]{inputenc}
\usepackage[T1]{fontenc}

%!%!%!%!%!%!%
% settings  %
%!%!%!%!%!%!%

%! Select the classyslides theme.
\usetheme{classyslides}

%! Set the accent color and highlight color to a greenish variant.
\definecolor{accent-color}{RGB}{5, 185, 125}
\definecolor{highlight-color}{RGB}{5, 245, 185}

%
%! Document meta data.
%
\title{There Is No Largest Prime Number}
\author[Euclid]{Euclid of Alexandria \\ \texttt{euclid@alexandria.edu}}
\date[ISPN �80]{27th International Symposium of Prime Numbers}

%! Presentation mode.
\mode<presentation>

%!%!%!%!%!%!%
% document  %
%!%!%!%!%!%!%
\begin{document}

\begin{frame}
\titlepage
\end{frame}

%! Table of contents.

\contentpagetrue
\begin{frame}
\frametitle{Outline}
\tableofcontents
\end{frame}

\section{Motivation}
\subsection{The Basic Problem That We Studied}
\contentpagefalse
\frame{\sectionpage}

\contentpagetrue
\begin{frame}
\frametitle{What Are Prime Numbers?}
\begin{definition}
A \emph{prime number} is a number that has exactly two divisors.
\end{definition}
\begin{example}
\begin{itemize}
\item 2 is prime (two divisors: 1 and 2).
\item 3 is prime (two divisors: 1 and 3).
\item 4 is not prime (\alert{three} divisors: 1, 2, and 4).
\end{itemize}
\end{example}
\end{frame}

\begin{frame}[fragile]
\frametitle{There Is No Largest Prime Number6}%
\framesubtitle{The proof uses reductio ad absurdum}%
\begin{theorem}
	There is no largest prime number.
	\tcblower
	\textbf{Proof}
	\begin{enumerate}
		\item<1-> Suppose $p$ were the largest prime number.
		\item<2-> Let $q$ be the product of the first $p$ numbers.
		\item<3-> Then $q + 1$ is not divisible by any of them.
		\item<4-> But $q + 1$ is greater than $1$, thus divisible by some prime number not in the first $p$ numbers.\qedhere
	\end{enumerate}   
\end{theorem}
\uncover<4->{The proof used \textit{reductio ad absurdum}.}
\end{frame}

\begin{frame}[t]
\frametitle{What's Still To Do?}
\begin{itemize}
	\item Answered Questions
		\begin{itemize}
			\item How many primes are there?
		\end{itemize}
	\item Open Questions
		\begin{itemize}
			\item Is every even number the sum of two primes?
		\end{itemize}
	\end{itemize}
\end{frame}

\begin{frame}[fragile]
\frametitle{An Algorithm For Finding Prime Numbers.}
\begin{Code*}[title={Finding Prime Numbers}]{%
	language=c%
	, basicstyle=\scriptsize\inconsolatafamily%
	, keywordstyle=\scriptsize\inconsolatafamily\bfseries%
}
int main (void)
{
	std::vector<bool> is_prime (100, true);
	for (int i = 2; i < 100; i++)
		if (is_prime[i])
		{
			std::cout << i << " ";
			for (int j = i; j < 100; is_prime [j] = false, j+=i);
		}
	return 0;
}
\end{Code*}
\begin{uncoverenv}<2>
Note the use of \verb|std::|.
\end{uncoverenv}
\end{frame}

\end{document}